\section{Summary}

% \begin{frame}{Summary of the Paper}
% \begin{itemize}
%     \item \textbf{Research Contributions}:
%     \begin{itemize}
%         \item Proposed a two-stage movie dubbing method that significantly enhances the consistency in timbre, emotional expression, and duration alignment of dubbing through multi-task pre-training and dubbing training.
%         \item The Prosody Consistency Learning (PCL) module improves emotional consistency in dubbing by aligning emotional features with phoneme-level prosody attributes.
%         \item The Duration Consistency Reasoning (DCR) module achieves precise duration matching between dubbing and video by aligning lip motion features with phoneme features.
%     \end{itemize}
%     \item \textbf{Technical Innovations}:
%     \begin{itemize}
%         \item Multi-Task Speaker Pre-training (MTSP) integrates Text-to-Speech (TTS) and Masked Language Model (MLM) tasks, significantly enhancing the model's pronunciation quality and contextual understanding.
%         \item Introduced Timbre-Adaptive Layer Normalization (TALN), which integrates timbre features into phoneme encoding and mel-spectrogram generation by predicting gain and bias for the input vector sequence, ensuring timbre consistency.
%     \end{itemize}
%     \item \textbf{Experimental Results}:
%     \begin{itemize}
%         \item The method outperforms current state-of-the-art methods on multiple metrics across the V2C-Animation and GRID benchmark datasets.
%     \end{itemize}
% \end{itemize}
% \end{frame}
\begin{frame}{Summary of the Paper}
\begin{itemize}
    \item \textbf{Research Contributions}:
    \begin{itemize}
        \item Proposed a two-stage movie dubbing method enhancing timbre, emotion, and duration alignment via MTSP and dubbing training.
        \item PCL module improves emotional consistency by aligning emotions with phoneme-level prosody.
        \item DCR module achieves precise duration matching by aligning lip motion with phoneme features.
    \end{itemize}
    \item \textbf{Technical Innovations}:
    \begin{itemize}
        \item MTSP integrates TTS and MLM tasks, improving pronunciation quality and contextual understanding.
        \item TALN integrates timbre features into phoneme encoding and mel-spectrogram generation, ensuring timbre consistency.
    \end{itemize}
    \item \textbf{Experimental Results}:
    \begin{itemize}
        \item Method outperforms state-of-the-art methods on V2C-Animation and GRID datasets.
    \end{itemize}
\end{itemize}
\end{frame}
% Second page: Reflections on the paper
% \begin{frame}{Reflections on the Paper}
% \begin{itemize}
%     \item \textbf{Learning Method from 'Simple' to 'Complex'}:
%     \begin{itemize}
%         \item The paper employs a two-stage learning framework of "from baby to dubber." It first pre-trains on a large-scale text-to-speech dataset to teach the model clear pronunciation and then fine-tunes on movie dubbing data to adapt to emotional and lip-sync alignment.
%         \item This method effectively addresses the scarcity and complexity of movie dubbing data, significantly improving dubbing quality and consistency.
%     \end{itemize}
%     \item \textbf{Implications for Practical Applications}:
%     \begin{itemize}
%         \item Offers new ideas for data scarcity issues: pre-training on general data followed by fine-tuning on specialized data can enhance model performance.
%         \item Provides insights for multimodal alignment tasks, such as virtual anchors and intelligent customer service.
%     \end{itemize}
%     \item \textbf{Future Research Directions}:
%     \begin{itemize}
%         \item Optimize the two-stage learning framework, such as introducing more modalities like gestures and facial expressions during the pre-training phase.
%         \item Introduce user feedback mechanisms in model training to further optimize dubbing effects.
%     \end{itemize}
% \end{itemize}
% \end{frame}
\begin{frame}{Reflections on the Paper}
\begin{itemize}
    \item \textbf{Learning Method from 'Simple' to 'Complex'}:
    \begin{itemize}
        \item Uses a two-stage framework: pre-trains on TTS data for clear pronunciation, then fine-tunes on dubbing data for emotion and lip-sync alignment.
        \item Addresses data scarcity and complexity, improving dubbing quality.
    \end{itemize}
    \item \textbf{Implications for Practical Applications}:
    \begin{itemize}
        \item Pre-training on general data followed by fine-tuning on specialized data enhances model performance.
        \item Provides insights for multimodal alignment tasks like virtual anchors and customer service.
    \end{itemize}
    \item \textbf{Future Research Directions}:
    \begin{itemize}
        \item Optimize two-stage learning by adding more modalities (e.g., gestures, facial expressions) during pre-training.
        \item Introduce user feedback mechanisms to further optimize dubbing effects.
    \end{itemize}
\end{itemize}
\end{frame}
\begin{frame}
\begin{itemize}
    \item References:
\end{itemize}
    \scriptsize
    \bibliography{ref}
    % % If there are too many references, you can adjust the font size as follows:
    % \tiny\bibliographystyle{apalike}
\end{frame}

\begin{frame}
    \begin{center}
        {\Huge\calligra Thanks!} \\
        \vspace{1cm} % Add some vertical space
    \end{center}
\end{frame}